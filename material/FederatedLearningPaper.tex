% Template for ICASSP-2021 paper; to be used with:
%          spconf.sty  - ICASSP/ICIP LaTeX style file, and
%          IEEEbib.bst - IEEE bibliography style file.
% --------------------------------------------------------------------------
\documentclass{article}
\usepackage{spconf,amsmath,bm,graphicx}
\usepackage[dvipsnames]{xcolor}
\usepackage{xcolor}   
\usepackage{tikz}
\usetikzlibrary{positioning}
\usetikzlibrary{shapes.symbols}
\usetikzlibrary{shapes}

%\include{mlbpbook_macros}
 
\usepackage{fontawesome5}

\usetikzlibrary{shadows}

\usepackage{algorithm}
\usepackage{algorithmic}
\usepackage{subcaption}
\usepackage[export]{adjustbox}


%% Math fonts, symbols, and formatting; these are usually needed
\usepackage{amsfonts,amssymb,amsbsy}
%\DeclareMathOperator*{\argmax}{arg\,max}
%\DeclareMathOperator*{\argmin}{arg\,min}

\definecolor{pinegreen}{cmyk}{0.92,0,0.59,0.25}
\definecolor{royalblue}{cmyk}{1,0.50,0,0}
\definecolor{lavander}{cmyk}{0,0.48,0,0}
\definecolor{violet}{cmyk}{0.79,0.88,0,0}

\tikzstyle{ncyan}=[circle, draw=cyan!70, thin, fill=white, scale=0.8, font=\fontsize{11}{0}\selectfont]
\tikzstyle{ngreen}=[circle,  draw=green!70, thin, fill=white, scale=0.8, font=\fontsize{11}{0}\selectfont]
\tikzstyle{nred}=[circle, draw=red!70, thin, fill=white, scale=0.8, font=\fontsize{11}{0}\selectfont]
\tikzstyle{ngray}=[circle, draw=gray!70, thin, fill=white, scale=0.55, font=\fontsize{14}{0}\selectfont]
\tikzstyle{nyellow}=[circle, draw=yellow!70, thin, fill=white, scale=0.55, font=\fontsize{14}{0}\selectfont]
\tikzstyle{norange}=[circle,  draw=orange!70, thin, fill=white, scale=0.55, font=\fontsize{10}{0}\selectfont]
\tikzstyle{npurple}=[circle,draw=purple!70, thin, fill=white, scale=0.55, font=\fontsize{10}{0}\selectfont]
\tikzstyle{nblue}=[circle, draw=blue!70, thin, fill=white, scale=0.55, font=\fontsize{10}{0}\selectfont]
\tikzstyle{nteal}=[circle,draw=teal!70, thin, fill=white, scale=0.55, font=\fontsize{10}{0}\selectfont]
\tikzstyle{nviolet}=[circle, draw=violet!70, thin, fill=white, scale=0.55, font=\fontsize{10}{0}\selectfont]
\tikzstyle{qgre}=[rectangle, draw, thin,fill=green!20, scale=0.8]
\tikzstyle{rpath}=[ultra thick, red, opacity=0.4]
\tikzstyle{legend_isps}=[rectangle, rounded corners, thin,fill=gray!20, text=blue, draw]

% Example definitions.
% --------------------
\def\x{{\mathbf x}}
\def\L{{\cal L}}

% Title.
% ------
\title{Template for Student Project for \\ 
	CS-E4740 Federated Learning}
%
% Single address.
% ---------------
\name{N.N. }
\address{Aalto University, Espoo, Finland}
%
% For example:
% ------------
%\address{School\\
%	Department\\
%	Address}
%
% Two addresses (uncomment and modify for two-address case).
% ----------------------------------------------------------
%\twoauthors
%  {A. Author-one, B. Author-two\sthanks{Thanks to XYZ agency for funding.}}
%	{School A-B\\
%	Department A-B\\
%	Address A-B}
%  {C. Author-three, D. Author-four\sthanks{The fourth author performed the work
%	while at ...}}
%	{School C-D\\
%	Department C-D\\
%	Address C-D}
%
\begin{document}
%\ninept
%
\maketitle
%
\begin{abstract}
This paper studies federated learning methods for high-precision weather forecasting. 
\end{abstract}
%
\begin{keywords}
Federated Learning, Networks, Personalized Machine Learning, Trustworthy AI 
\end{keywords}
%
\section{Introduction}
\label{sec:intro}
\begin{itemize}
\item Explain the background (real-life scenario) of your ML application (see \cite[Ch. 2]{MLBasics}).  
\item Summarize the relevant literature (state-of-the art). 
\item Briefly outline the structure of this paper. 
\end{itemize}


\section{Problem Formulation}
\label{sec:pf}

\begin{itemize} 
\item Formulate your application as an instance of GTV minimization \cite[Sec. 7]{lecnotesfl} 
\item Discuss you choice/construction of the empirical graph whose nodes carry local datasets and local models. 
\item Provide a precise definition of the local datasets, their data points, features and labels. 
\item Explain the source of the data used in your project. 
\end{itemize} 

\section{Methods} 
\label{sec_methods} 

\begin{itemize} 
\item Clearly state the number of datapoints in each local dataset. 
\item Mention any specific characteristics of data and clearly state if any data preprocessing has been implemented.
\item Explain your feature selection process (no theoretical justification needed).
\item Describe and explain (why?) your choice of local models. 
\item Describe and explain which FL algorithm you have used to train the local models \cite[Sec. 9]{lecnotesfl}
\item Describe and explain (why?) your choice of loss function(s), e.g., logistic loss
\item Explain the process of model validation - how did you split the data into training, validation and test sets. What are the sizes of each set and why did
you make such design choice.
\end{itemize} 

\section{Results} 
\label{sec_results} 

\begin{itemize} 
	\item Compare and discuss the training and validation errors obtained for each node of the empirical graph. 
	\item Which is the final chosen method and why?
	\item What is the test error (for each node) of the final chosen method?
\end{itemize} 

\section{Conclusion}

\begin{itemize} 
\item Provide a succinct summary of your findings.
\item Are the results suggesting that the problem is solved satisfactorily, or might
there be room for improvement?
\item Ponder about possible limitation of the methods and how it can be further improved.
\end{itemize} 





\bibliographystyle{IEEEbib}
\bibliography{Literature}


\end{document}
